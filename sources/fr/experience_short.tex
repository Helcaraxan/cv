\vspace{-0.3cm}
\section{Expérience}
\cventry{2013--2016}{Enseignant vacataire}{Université Grenoble Alpes}{Grenoble}{}{
	Enseignant et assistant pour plusieurs cours de License et de Master à l'Université Grenoble Alpes
	ainsi qu'à l'ENSIMAG.
	\begin{footnotesize}
		\begin{itemize}
			\item Introduction à la programmation et à l'algorithimique
			\item Algorithmique avancée
			\item Architecture de processeurs
			\item Projets de fin d'année
		\end{itemize}
	\end{footnotesize}
}

\cventry{2013--2016}{Ingénieur R\&D}{Kalray}{Grenoble}{}{
	Poste d'ingénieur combiné à une thèse par financement CIFRE avec l'Inria.
	\begin{footnotesize}
		\begin{itemize}
			\item R\&D pour la chaîne de compilation de Kalray (GCC, LLVM, \dots)
			\item Conception et spécification du ``réseau-sur-puce'' (NoC) de Kalray
			\item Suivi et développement pour des projets collaboratifs
			\item \'E{}criture et présentation d'articles scientifiques
		\end{itemize}
	\end{footnotesize}
}

\cventry{2012\\ Mars--Juin}{Stage de Master}{LIG}{Grenoble}{}{
	Travail sur le Berkeley Open Infrastructure for Network Computing (BOINC).
	\begin{footnotesize}
		\begin{itemize}
			\item Recherche \& implémentation de nouvelles politiques d'ordonnancements pour les projets
				de petite à moyenne taille
		\end{itemize}
	\end{footnotesize}
}

\cventry{2011\\ Mars--Août}{Stage de Master}{Kalray S.A}{Grenoble}{}{
	Stage en R\&D.
	\begin{footnotesize}
		\begin{itemize}
			\item Implémentation spécifique à l'architecture de Kalray du protocole AES
			\item \'E{}valuation des conceptions architecturales en développement
			\item Prototypage de la pile logicielle TCP/IP
		\end{itemize}
	\end{footnotesize}
}

\cventry{2010\\ Juin--Juillet}{Stage de License}{Inria}{Grenoble}{}{
	Travail sur la modélisation et la prédiction du trafic routier.
	\begin{footnotesize}
		\begin{itemize}
			\item Extension d'un model mathématique de flux
			\item Prototypage d'une implémentation taille réelle d'un simulateur de flux
		\end{itemize}
	\end{footnotesize}
}
