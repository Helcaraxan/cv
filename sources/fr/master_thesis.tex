\vspace{-0.3cm}
\section{Master thesis'}
\cvline{Titre}{\emph{Ordonnancement du Calcul Distribué dans l'Infrastructure BOINC}}
\cvline{Responsable}{A. Legrand}
\cvline{Description}{\small Dans la course aux performances dans les calculs à grande échelle la
	puissance inutilisé des ordinateurs personnels a pendant longtemps été une resource peu exploité.
	Les initiatives de Calcul Distribué tels que le project BOINC ont entrepris de mettre à profit
	cette resource. Cependant l'évolution des types d'utilisation qui en est faite nécessite de
	retravailler l'ordonnancement des tâches de calcul sur l'ensemble des clients. Les principes comme
	la distribution de resources équitable, la complétion de groupes de tâches et la spécification de
	dates limites sont au c\oe{}ur de cette étude.
}

\cvline{Titre}{\emph{Cryptography Haute-performance à travers une Implémentation \og{}Bitslice\fg{}}}
\cvline{Responsable}{B. Dupont de D\^{i}nechin}
\cvline{Description}{\small La bande-passante toujous croissante des connections réseau crée des
	défis pour le traitement cryptographic à haute vitesse et parallelisé. L'utilisation
	d'instructions dédiées permet à des implémentations \og{}bitslice\fg{} des protocols tels l'AES de
	tirer parti de la nouvelle vague d'architecture manyc\oe{}urs. Ce travail présente une étude de
	cas à la fois théorique et pratique d'une telle implémentation.
}
